\documentclass{scrartcl}
\usepackage{hyperref}
\usepackage{times}

% usepackage definitions for KiekerTutorial.tex and for roadmap.tex
\usepackage{hyperref}
\newcommand{\goodgap}{\hspace{\subfigtopskip}\hspace{\subfigbottomskip}}
\usepackage{times}
\usepackage[latin1]{inputenc}
\usepackage{graphicx}
\usepackage{subfigure}
\usepackage{natbib}
\usepackage{listings}
\listfiles \lstloadlanguages{java} 
\lstset{language=java,
  showspaces=false,
  escapeinside={/*@}{@*/},
  showstringspaces=false,
  numbers=left,
  numberstyle=\tiny,
  numbersep=5pt,
  breaklines=true,
  breakautoindent=true,
  float=hbp,
  basicstyle=\ttfamily\small,
  columns=flexible,
  tabsize=2,
  extendedchars=true,
  frame=trbl
}
\newcommand{\todobox}[1]{\marginpar{\footnotesize \textbf{ TODO:  #1 }}}
\newcommand{\structurebox}[1]{%
	\par\noindent%
        \begin{center}
        \doublebox{\parbox{.8\columnwidth}{\texttt{Structure guide:} #1}}%
        \end{center}
}
\newcommand{\tpmon}{\textit{Tpmon}}
\newcommand{\tpan}{\textit{Tpan}}
\newcommand{\kieker}{\textit{Kieker}}
\newcommand{\tpmonltw}{\textit{tpmonltw.jar}}
\newcommand{\aspectjweaverjar}{\textit{aspectjweaver.jar}}



\newcommand{\kiekerurl}[1]{\small\url{http://www.matthias-rohr.com/kieker/#1}\normalsize}
\newcommand{\kiekertutorialurl}{\kiekerurl{KiekerTutorial.pdf}}
\newcommand{\kiekerroadmapurl}{\kiekerurl{KiekerTutorial.pdf}}

\title{\kieker -- Roadmap}
 \date{\today \\ \kiekerurl{roadmap.pdf}}
 \author{Matthias Rohr, Andr\'{e} van Hoorn, Nina S. Marwede}


\begin{document}
\maketitle
%\tableofcontents
\noindent
This document describes the changes in current and old releases and outlines plans for future releases of Kieker's monitoring component \tpmon{}. Kieker is an open source project
for the monitoring and visualization of Java applications. It's major target are Java (Web) applications. Details can be found in the tutorial\footnote{\kiekertutorialurl}.

\

\noindent \large{\textbf{Current and past releases overview:}}
\begin{center}%\vspace{-3mm}
 \begin{tabular}{|l|l|p{8cm}|} \hline
\textbf{Version number} & \textbf{Intended date} & \textbf{Major feature} \\ \hline
0.91 & 04/2009 & Completely re-engineered and re-documented (tutorial) release. Aspects are now in @Aspect-style and use ThreadLocal. Performance was also improved. Support for monitoring in the spring framework was extended. Derby db can be used. Distributed monitoring works.\\ \hline
0.2 & 06/2008 & Improved dependability \\ \hline
0.1 & 11/2007 & Initial release with basic command line functionality \\ \hline
\end{tabular}
\end{center}

\noindent \large{\textbf{Future releases:}}
\begin{center}%\vspace{-3mm}
 \begin{tabular}{|l|l|p{8cm}|} \hline
\textbf{Version number} & \textbf{Intended date} & \textbf{Major feature} \\ \hline

0.95 & 07/2009 & Tpan first release (maybe as web service as well) for generating calling dependency graphs ,JMS support\\ \hline
0.98 & 10/2009 & Usability improvements; graphical user interface; server-side runtime model; support for cloud computing.\\ \hline
\end{tabular}
\end{center}


\section{Changes in current release}

\subsection{Release 0.91}
\begin{enumerate}
\item The data type of threadid was changed to long. This heavily reduces the amount of memory and computation required for post-processing.
\item From now on, Tpmon uses  @Aspect-style for the aspects (cross-cutting code) instead of the old-style
    .aj AspectJ files. The advantage of the @Aspect-style is that the aspect files can be directly handled by Java IDEs and Compilers.
\item The data structure KiekerExecutionRecord now is used all the way from the measurement point, through the controller to the selected monitoring data writer and queues of writers that use producer-consumer pattern. This avoids type-casting and improves consistency through the architecture.
\item  A new property allows the user to select the monitoring writer: \\
    \verb.monitoringDataWriter=SyncFS|AsyncFS|SyncDB|AsyncDB. \\
The value of this property can be either one of the constants shown below or a full-qualified
classname of a class (implementing kieker.tpmon.IMonitoringDataWriter) loaded
dynamically during runtime (must be in the runtime classpath). Classes loaded by
the classname are constructed by calling the default constructor followed by a
called to the method init(String monitoringDataWriterInitString) with the
value of the below-listed property monitoringDataWriterInitString.

Existing constants are:
\begin{itemize}
\item SyncFS (Synchronous File System Writer)
\item AsyncFS (Asynchronous File System Writer)
\item SyncDB (Synchronous Database Writer)
\item AsyncDB (Asynchronous Database Writer)
\end{itemize}
The old properties to activate asynchr. FS and DB respectively are removed. Switching to
\tpmon{} 0.91 requires to replace old tpmon.properties wit a new one based on tpmon.properties.examples of 0.91.

\item The experimental JMS Writer prototype is not anymore part of the SourceForge distribution of \tpmon{}.
\end{enumerate}


\section{Changes in old releases}

\subsection{Release 0.2 - "Improved Dependability for Tpmon" 01/2008}
\begin{enumerate}
%  \item Visualization
\item Input
\begin{enumerate}
 \item Asynchronous access to Tpmon database (schema 0.7+)
\item On-the-fly conversion into Message traces
\end{enumerate}
\item Quality
\begin{itemize}
 \item Additional test cases
\item One month test run
\end{itemize}
% \item User documetation
% \item Developer Documentation
\end{enumerate}

\subsection{Release 0.1 -- ``Initial release'' 11/2007}
% \begin{enumerate}
%  \item Visualization
% \item Output
% \item Input
% \item Quality
% \item User documetation
% \item Developer Documentation
% \end{enumerate}

\begin{enumerate}
\item Visualization
\begin{enumerate}
 \item UML Sequence Diagrams for single Message Trace (generates code for pic2plot)
\item Markov Chain from Collection of Message Traces (generates GraphViz code)
\item Dependency Graphs from Collections of Message Traces (generates code for GraphViz)
\end{enumerate}
\item Output (Code generation for external tools)
\begin{enumerate}
 \item Storing visualization code in /tmp (or other path configured)
\item Showing in console command to create image file from visualization code
\end{enumerate}
\item Input
\begin{enumerate}
\item Synchronous access to Tpmon database (schema 0.7+)
\item Access traces by experimentid, traceid, tin, operation, sessionid
\end{enumerate}
\item Quality
\begin{enumerate}
 \item Complete basic JavaDoc
\item Testing
\begin{enumerate}
 \item Datainput
\item Several unit tests for the core of Tpan (creation of message traces and Execution Traces)
\item One unit test for each plugin
\item One end2end unit test testing the 4 core plugins and dataimport
\end{enumerate}
\item Handle fault scenario "no database driver"
\item Handle fault scenario "database username or password wrong?"
\end{enumerate}
\item User documentation
\begin{enumerate}
 \item Tutorial Tpmon
\item Tutorial Tpan
\item Manual Tpmon
\item Use cases
\item Alternative configurations
\item Manual Tpan
\end{enumerate}

\item Developer Documentation
\begin{enumerate}
\item Developer manual Tpmon
\item Developer manual Tpan
\item Roadmap
\end{enumerate}
\end{enumerate}

\end{document}
