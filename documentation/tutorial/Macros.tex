% If the name ``Kieker'' (or the design of the name in the tutorial) should be replaced sometime.
\newcommand{\Kieker}{\textit{Kieker}}
% If the names for the parts of Kieker change.
\newcommand{\KiekerMonitoring}{Tpmon}
\newcommand{\KiekerAnalysis}{Tpan}
% The complete url where to download Kieker.
\newcommand{\KiekerDownloadUrl}{\url{http://sourceforge.net/projects/kieker/files}}

% The current version-string.
\newcommand{\version}{1.1-refactoring-branch}

% The name of the .jar-files from Kieker.
\newcommand{\analysisCtrlJar}{kieker-analysis-\version\_ctrl.jar}
\newcommand{\monitoringCtrlJar}{kieker-monitoring-\version\_ctrl.jar}
\newcommand{\monitoringLtwJar}{kieker-monitoring-\version\_ltw.jar}
\newcommand{\commonJar}{kieker-common-\version.jar  }
\newcommand{\toolsjar}{kieker-tools-\version.jar}
\newcommand{\monitoringCtwJar}{kieker-monitoring-\version\_ctw.jar}
\newcommand{\monitoringControlServletWar}{kieker-monitoring-\version\_controlServlet.war}

% The following commands set the listings for the different (programming) languages correctly.
% For the first they use all the same settings.
\newcommand{\setListing}[1]{
\lstset{
language=#1,                
basicstyle=\tiny,       	% the size of the fonts that are used for the code
showspaces=false,               % show spaces adding particular underscores
showstringspaces=false,         % underline spaces within strings
showtabs=false,                 % show tabs within strings adding particular underscores
frame=single,	                % adds a frame around the code
tabsize=2,	                % sets default tabsize to 2 spaces
captionpos=b,                   % sets the caption-position to bottom
breaklines=true,                % sets automatic line breaking
breakatwhitespace=false,        % sets if automatic breaks should only happen at whitespace
title=\lstname,                 % show the filename of files included with \lstinputlisting; also try caption instead of title
}
}
\newcommand{\setJavaCodeListing}{\setListing{Java}}
\newcommand{\setBashListing}{\setListing{Bash}}
\newcommand{\setXMLListing}{\setListing{XML}}
