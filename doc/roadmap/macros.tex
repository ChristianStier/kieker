%%%%%%%%%%%%%%%%%%%%%%%%%%%%%%%%%%%
% macors.tex
%
% author: Reiner Jung
%

%% einstellung f�r mehrere grafiken auf einer seite
%
%
%
\renewcommand{\textfraction}{0.05} 
\renewcommand{\topfraction}{0.95} 
\renewcommand{\bottomfraction}{0.95} 
\setcounter{topnumber}{4} 
\setcounter{totalnumber}{5} 
 



%% setzt pfad f�r bilder
%
% Parameter:
%   Neuer Pfad f�r Bilder
%
% initialisierung
\newcommand{\img}{images}
% 
\newcommand{\setimage}[1]{\renewcommand{\img}{#1/images}}

%% setzt pfad f�r subdokumente
%
% Parameter:
%   Neuer Pfad f�r Bilder
%
% initialisierung
\newcommand{\docdir}{.}
% 
\newcommand{\setdocdir}[1]{\renewcommand{\docdir}{#1}}

%% input mit docdir usage
%
% Parameter:
%   Datei die eingebunden werden soll
\newcommand{\use}[1]{\input{\docdir/#1}}

%% \includepicture
% Erlaubt es ein Bild in ein Dokument einzuf�gen
%
% Parameter:
%   1. Werte f�r \includegraphics z.B. die Breite width=11.7cm
%   2. Dateiname der eps-Datei im Unterverzeichnis graphics
%   3. Bildunterschrift
%   4. Label/Referenzname

\newcommand{\incpic}[4]{
\begin{figure}[ht]
\begin{center}
\fbox{
\includegraphics[#1]{\img/#2}}
\caption{#3} \label{#4}
\end{center}
\end{figure}}

%% \includepicture
% Erlaubt es ein Bildschirmabbild in ein Dokument einzuf�gen
% Erzeugt keinen Rahmen
%
% Parameter:
%   1. Werte f�r \includegraphics z.B. die Breite width=11.7cm
%   2. Dateiname der eps-Datei im Unterverzeichnis graphics
%   3. Bildunterschrift
%   4. Label/Referenzname

\newcommand{\incscreen}[4]{
\begin{figure}[hpt]
\begin{center}
\includegraphics[#1]{\img/#2}
\caption{#3} \label{#4}
\end{center}
\end{figure}}


%% \figref
% reference a figure
\newcommand{\figref}[1]{(see fig. \ref{#1})}

%% \secref
% reference a section
\newcommand{\secref}[1]{(see section \ref{#1})}


%% \note
% Erzeugt eine Notiz im Text, welche sp�ter noch entfernt werden muss.
% (von Michael um Autor erweitert, da das auch ganz interessant sein k�nnte...) 
%
% Parameter:
%   1. Autor der Notiz
%   2. Die Notiz

\newcommand{\note}[1]{\paragraph{Note:} \textit{#1} \vskip 2em}

%% action
\newcommand{\actionpre}{}
\newcommand{\actionpost}{}
\newcommand{\actionrole}{}

\newcounter{actioncount}
\setcounter{actioncount}{1}

\newenvironment{action}[1]{
\renewcommand{\actionpre}{}
\renewcommand{\actionpost}{}
\renewcommand{\actionrole}{}
\vspace{3mm}
\noindent
\vline\hspace{2mm}
\begin{minipage}{0.95\textwidth}
\paragraph{Action \arabic{actioncount}: #1}
}{
\begin{description}
\ifthenelse{\equal{\actionrole}{}}{}{\item[Roles:] \actionrole}
\ifthenelse{\equal{\actionpre}{}}{}{\item[Precondition:] \actionpre}
\ifthenelse{\equal{\actionpost}{}}{}{\item[Postcondition:] \actionpost}
\end{description}\end{minipage}\\
\vspace{3mm}\stepcounter{actioncount}}

%% \pre
\newcommand{\pre}[1]{\renewcommand{\actionpre}{#1}}

%% \post
\newcommand{\post}[1]{\renewcommand{\actionpost}{#1}}

%% \role
\newcommand{\role}[1]{\renewcommand{\actionrole}{#1}}

%%
\newcounter{reqcount}
\setcounter{reqcount}{1}
\newcommand{\req}[2]{\paragraph{Requirement \arabic{reqcount}: #1} #2\stepcounter{reqcount}}

%% OWL
\lstdefinelanguage{OWL}
{keywords={owl:AllDifferent,
allValuesFrom,
AnnotationProperty,
backwardCompatibleWith,
cardinality,
Class,
complementOf,
DataRange,
DatatypeProperty,
DeprecatedClass,
DeprecatedProperty,
differentFrom,
disjointWith,
distinctMembers,
equivalentClass,
equivalentProperty,FunctionalProperty,hasValue,imports,incompatibleWith,
intersectionOf,
InverseFunctionalProperty,
inverseOf,
maxCardinality,
minCardinality,
Nothing,
ObjectProperty,
oneOf,
onProperty,
Ontology,
OntologyProperty,
priorVersion,
Restriction,
sameAs,
someValuesFrom,
SymmetricProperty,
Thing,
TransitiveProperty,
unionOf,
versionInfo,
about,
ID,
parseType,
resource,
type,
subClassOf,
owl,rdf,rdfs}, keywordsprefix={owl:,rdf:,rdfs:}, morecomment=[s]{<!--}{-->}, sensitive=true,
}

\lstset{tabsize=4, basicstyle=\sffamily\small, keywordstyle=\bfseries, columns=[r]fullflexible}

\definecolor{keywordA}{RGB}{127,24,156}
\definecolor{keywordB}{RGB}{50,50,255}
\definecolor{keywordC}{RGB}{255,50,50}

\lstdefinelanguage{XTEND}{
alsoletter={:->},
morekeywords=[1]{create,::,:,if,else,then},
keywords=[2]{->},
keywords=[3]{this,null},
keywords=[4]{Integer,List},
keywordstyle=[1]\color{keywordA},
keywordstyle=[2]\color{keywordB},
keywordstyle=[3]\color{keywordB},
keywordstyle=[4]\color{keywordC},
stringstyle=\bfseries,
string=[b]{\"},
}

%% \partner
%
% fields can be empty
%
% 1 = name
% 2 = strasse
% 3 = plz + ort
% 4 = tel
% 5 = fax
% 6 = mail
% 7 = www
\newcommand{\partner}[7]{
\pagebreak{}

\section*{Kontakt}

\vbox{
\hbox{MENGES-Projektpartner\hfill}
\vskip2mm
\hbox{\textbf{#1}\hfill}
\vskip2mm
\hbox{#2\hfill}
\hbox{#3\hfill}
\vskip1mm
\hbox{Tel.: #4}
\hbox{Fax: #5}
\vskip1mm
\hbox{E-Mail: \href{mailto:#6}{\texttt{#6}}}
\hbox{Web: \url{#7}}
}
}
