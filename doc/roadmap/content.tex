%%%%%%%%%%%%%%%%%%%%%%%%%%%%%%%%%%%%%%%%%%
%%
%% add content here


%%
\section{Ziele}

\begin{itemize}
\item Instrumentierung für verschiedene Programmiersprachen via Source-Code, Code-Injection, Intermediate-Code
\item Sammlung von Auswertungsverfahren
\item Sammlung von Visualisierungen
\item Modellgetriebene Instrumentierung
\item Klare Schnittstellen und Methodiken für alle Komponenten von Kieker
\end{itemize}

%%
\section{Bestehende Konzepte}

%
\subsection{Kieker 1.2}

Kieker-Basis-Projekt mit der klassischen Kieker Instrumentierungstechnik für Java und der Probe, Storage, Auswertungs-Architektur.

%
\subsection{Monitoring Tool}

Jens hat im Rahmen seiner Dis. ein grafisches Werkzeug gebaut, welches ebenfalls zum Kieker-Kontext gehört.

\begin{itemize}
\item Graphischer Editor für die Kombination von Filtern/Analyseplugins
\item Visualisierung der Ergebnisse
\end{itemize}

%
\subsection{Kanalyse}

\begin{itemize}
\item Pipe-and-Filter-Sprache für die Kombination von Filtern und ggf. die Komposition von Filtern.
\item Ecore-Modell für Plugins
\end{itemize}

%%
\subsection{Projekt-Infrastruktur}

Um ein die einzelnen Projekte zusammenführen zu können und ein einheitliches Framework zu bekommen sind verschiedene Schritte notwendig. Für den Anfang schlage ich deshalb ein gemeinsames Repository für Kieker 2.0 vor in das wir dann die einzelnen Werkzeuge integrieren. Da wir alle hip und modern sind empfiehlt sich hier git zu benutzen.

Im Moment wird zwar Sourceforge benutzt, aber ein integriertes Ticket, Versions und Wiki System können wir auch mit Trac und Git realisieren. Ebenso können wir Hudson hier einbinden. Im Grunde entspricht dieses Setup auch dem Setup der RTSYS Gruppe von der wir dann auch entsprechend Technologietransfer haben können. Und ich glaube dieses Setup entspricht auch der generellen Strategie in der Abteilung.

%%
\section{Mögliche und Geplante Arbeiten}

Auf inhaltlicher Ebene sollten wir uns die einzelen Pakete ansehen und prüfen was davon weiter zu verwenden ist und vor allem wie es in ein Gesamtprojekt eingebracht werden kann ohne gleich alles neu schreiben zu müssen.

Nach einem solchen Bootstrapping, können wir uns dann um Detail kümmern, welche auch als Arbeiten für Studierende ausgelegt werden können. Im Einzelnen sind das:

\begin{itemize}
\item Generisches Analyse-Framework (Design) auf Basis von Jens' und Andrés Arbeiten
\item Metamodell für das Logging-Format sowie Transformationen, die aus Logging-Format-Modellen passende Serialisierer (Reader/Writer) erzeugen \fbox{BA}
\item Implementierung des Frameworks \fbox{mehrere Arbeiten}
\item Meta-Modell/Sprache für das Analyse-Framework \fbox{MA}
\item Graphischen Editor für das Analyse-Framework \fbox{BA/MA}
\item Instrumentierungsbeschreibung und Generator für Kieker-artigen Instrumentierungscode für Java \fbox{BA/MA}
\item Instrumentierungsbeschreibung und Generator für Kieker-artigen Instrumentierungscode für C\# \fbox{BA/MA}
\item Instrumentierungsbeschreibung und Generator für Kieker-artigen Instrumentierungscode für ST/FUP \fbox{BA/MA}
\end{itemize}

