% The name of Kieker, just for the case that the design of this should change.
\newcommand{\Kieker}{\textsf{Kieker}}

% The current version-string.
\newcommand{\version}{1.5-trunk}

% The single parts of Kieker and some files.
\newcommand{\KiekerMonitoringPart}{\textsf{Kieker.Monitoring}}
\newcommand{\KiekerAnalysisPart}{\textsf{Kieker.Analysis}}
\newcommand{\analysisJar}{kieker-analysis-\version.jar}
\newcommand{\monitoringJar}{kieker-monitoring-\version.jar}
\newcommand{\commonJar}{kieker-common-\version.jar}
\newcommand{\toolsJar}{kieker-tools-\version.jar}
\newcommand{\commonsLoggingJar}{commons-logging-1.1.1.jar}
\newcommand{\monitoringPropertiesFile}{kieker.monitoring.properties}
\newcommand{\analysisPropertiesFile}{kieker.analysis.properties}
\newcommand{\logFourJPropertiesFile}{log4j.properties}
\newcommand{\aopFile}{aop.xml}
\newcommand{\binaryFileForDownload}{kieker-\version{}\_binaries.zip}

% The complete url where to find Kieker.
\newcommand{\KiekerURL}{\url{http://sourceforge.net/projects/kieker/files}}

% This is how we call the kieker directory.
\newcommand{\KiekerDir}{kieker-\version{}}%{$<$KIEKER-DIR$>$}

% These commands are necessary to mark classes, methods and files within the document.
\newcommand{\class}[1]{\texttt{#1}}
\newcommand{\method}[1]{\textit{#1}}
\newcommand{\dir}[1]{\texttt{#1}}
\newcommand{\file}[1]{\texttt{#1}}

% This command formats the "new" files within a directory tree to get the users attention.
\newcommand{\newFilesDirTreeFormat}{\color{blue}}
\newcommand{\DirInDirTree}[1]{\textbf{#1}}

% These commands are for notifying the reader about something important.
\newcommand{\marginbox}[1]{\todo[noline]{#1}}
\newcommand{\notify}{\marginbox{\huge{\rightpointleft}}}
\newcommand{\warning}{\marginbox{\huge{\Stopsign}}}

% TODO command for our document
\newcommand{\TODO}[1]{\todo[inline,color=green!40]{TODO: #1}}

\makeatletter
\newcommand{\SYMBOLBOX}[2]{%
\begin{lrbox}{\@tempboxa}
\begin{minipage}[t]{0.1\textwidth}\vspace{1pt}
\begin{center}
\huge#1
\end{center}
\end{minipage}
\begin{minipage}[t]{0.8\textwidth}
#2%
\end{minipage}
\end{lrbox}
\todo[inline,bordercolor=black]{\usebox{\@tempboxa}}
}
\makeatother

\newcommand{\NOTIFYBOX}[1]{\SYMBOLBOX{\leftpointright}{#1}}
\newcommand{\WARNBOX}[1]{\SYMBOLBOX{\Stopsign}{#1}}

% This color will be used as backgroundcolor for the listings.
\definecolor{verylightgray}{gray}{.95}

% The following commands set the listings for the different (programming) languages correctly.
% For the first they use all nearly the same settings.
\newcommand{\setListing}[4]{
\lstset{
language=#1,          
numbers=#2,
basicstyle=#3,       	% the size of the fonts that are used for the code
showspaces=false,               % show spaces adding particular underscores
showstringspaces=false,         % underline spaces within strings
showtabs=false,                 % show tabs within strings adding particular underscores
%frame=shadowbox,	                % adds a frame around the code
% frame=lrtb,
rulesepcolor=\color{black},
linewidth=0.98\columnwidth,
xleftmargin=1cm,
tabsize=2,	                % sets default tabsize to 2 spaces
captionpos=b,                   % sets the caption-position to bottom
breaklines=true,                % sets automatic line breaking
breakatwhitespace=false,        % sets if automatic breaks should only happen at whitespace
title=\lstname,                 % show the filename of files included with \lstinputlisting; also try caption instead of title
escapechar={#4},
backgroundcolor=\color{verylightgray}
}
}
\newcommand{\setJavaCodeListing}{
\setListing{Java}{left}{\sffamily\scriptsize}{\#}
}
\newcommand{\setBashListing}{\setListing{Bash}{none}{\sffamily\scriptsize}{\#}}
\newcommand{\setXMLListing}{\setListing{XML}{none}{\sffamily\scriptsize}{}}

% This is the definition for the environment, that can be used for the background of the dirtrees.
\makeatletter\newenvironment{graybox}{%
   \begin{lrbox}{\@tempboxa}\begin{minipage}{0.965\columnwidth}}{\end{minipage}\end{lrbox}%
   \fcolorbox{white}{verylightgray}{\usebox{\@tempboxa}}
}\makeatother
