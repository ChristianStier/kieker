\chapter{Kieker Tools}\label{chp:Kieker-Tools}
	\section{Trace Analysis}
		\subsection{Textual Trace and Equivalence Class Representations}
			\subsubsection{Execution Traces}
			\subsubsection{Message Traces}
			\subsubsection{Trace Equivalence Classes}
		\subsection{Sequence Diagrams}
			\subsubsection{Deployment-Level Sequence Diagrams}
			\subsubsection{Assembly-Level Sequence Diagram}
		\subsection{Call Trees}
			\subsubsection{Trace Call Trees}
			\subsubsection{Aggregated Call Trees}
		\subsection{Dependency Graphs}
			\subsubsection{Container Dependency Graphs}
			\subsubsection{Component Dependency Graphs}
			\subsubsection{Operation Dependency Graphs}
			\subsubsection{Response Times}
		\subsection{HTML Output of the System Model}	
			
	\section{Kieker WebGUI}\label{chp:Kieker-WebGUI}
			
		The \KiekerWebGUI{} is a JavaEE-based web application to assemble, control, and observe Kieker analyses. Although currently still in the beta state, it already provides a user and project management, a graphical editor, and a control interface for analyses. A cockpit, which can be used, for example, for the realtime monitoring of applications, is currently under development.

		Like \Kieker{}, the \KiekerWebGUI{} project is licensed under the Apache License, Version 2.0. 
		
		\subsection{Download and Installation}
		
			The application can be downloaded as \file{.zip} and \file{.tar.gz} file on the \Kieker{} website under \url{http://kieker-monitoring.net/download/}. Once downloaded and extracted, the directory structure in Figure~\ref{fig:webgui-binary-layout} should be visible. In order to start the web application on Jetty, a lightweight web server, execute the suitable start script for your operation system in the \file{bin} directory. Depending on the system, the start procedure can take several minutes. Once started, the web application is available under \url{http://localhost:8080/Kieker.WebGUI/login}.
			
			\begin{figure}[h!]
				\begin{graybox}
					\dirtree{%
						.1 \DirInDirTree{\KiekerWebGUIDir/}.
							.2 \DirInDirTree{bin/}\DTcomment{Start scripts for the \KiekerWebGUI{}}.
								.3 Kieker.WebGUI.bat.
								.3 Kieker.WebGUI.sh.
							.2 \DirInDirTree{lib/}\DTcomment{Libraries required to start the \KiekerWebGUI{}}.
								.3 \ldots.
							.2 \DirInDirTree{target/}.
								.3 Kieker.WebGUI-1.7.war\DTcomment{The web application archive containing the \KiekerWebGUI{}}.
					}
				\end{graybox}
				
				\caption{Directory structure and contents of \KiekerWebGUI{}'s binary distribution}
				\label{fig:webgui-binary-layout}
			\end{figure}
			
			\noindent
			The web application provides per default three users (Table~\ref{tab:webgui-default-users}), which can be used to log in. Further users can be created when logged in as administrator.
			
			\begin{table}[h!]
				\center
				
				\begin{tabular}{|c|c|}
					\hline
					Username & Password\\
					\hline
					\hline
					guest    & kieker\\
					user     & kieker\\
					admin    & kieker\\
					\hline
				\end{tabular}
			
				\caption{Default users in the \KiekerWebGUI{}}
				\label{tab:webgui-default-users}
			\end{table}
			
		\subsection{Quickstart Example}
		
			\NOTIFYBOX{
				For the quickstart example it is assumed that you are already logged into the \KiekerWebGUI{}, either as an user or as an administrator. You should also enable javascript and cookies, as both is necessary for the functionality of the application. 
			}
			
			\noindent
			After the log in, you see the main page for the project management (Figure~\ref{fig:webgui-project-management-page}). Create a new project by clicking \texttt{File $\to$ New Project}. Enter a name (\texttt{Timer-Example} e.g.) and click \texttt{ok}. The project should now appear in the list. A left click on the name offers various options. For the moment choose \texttt{Analysis Editor}. The browser should navigate to the analysis editor page (Figure~\ref{fig:webgui-analysis-editor-page}).
			
			\begin{figure}[h!]
				\caption{The project management page}
				\label{fig:webgui-project-management-page}
			\end{figure}

			\noindent
			In the analysis editor you should have multiple plugins in the left toolbar available. Search and add the components \texttt{TimeReader} and \texttt{TeeFilter} with a left click to the editor. They should appear in the center of the window. Move them around a little bit and connect the upper port of the \texttt{TimeReader} with the port of the \texttt{TeeFilter} by clicking on them. You should have a structure like in Figure~\ref{fig:webgui-analysis-example} on your screen. Save the project by using \texttt{File $\to$ Save Project}
		
			\begin{figure}[h!]
				\caption{The analysis editor page}
				\label{fig:webgui-analysis-editor-page}
			\end{figure}
			
			\begin{figure}[h!]
				\caption{A simple example}
				\label{fig:webgui-analysis-example}
			\end{figure}
			
			\noindent
			Now we start our simple analysis. You can change directly to the analysis controller page (Figure~\ref{fig:webgui-analysis-controller-page} ) by using the \texttt{Analysis} button at the top right of the page. Use the buttons \texttt{Instantiate Analysis} and \texttt{Start Analysis} at the bottom of the page to run the analysis. The result of the simple example can be seen in the console output of the web application. The \texttt{TimeReader} sends every second the current timestamp (in nano seconds) to the \texttt{TeeFilter} which prints it to the console output. An example output can be seen in Listing~\ref{lst:webgui-analysis-example}.
			
			
			\begin{figure}[h!]
				\caption{The analysis controller page}
				\label{fig:webgui-analysis-controller-page}
			\end{figure}
			
			\setTextListing
			\begin{lstlisting}[gobble = 8, label=lst:webgui-analysis-example, caption=Execution of the example analysis]
				TeeFilter(TimestampRecord) -1;211647816548852
				TeeFilter(TimestampRecord) -1;211648816610293
				TeeFilter(TimestampRecord) -1;211649816655733
				TeeFilter(TimestampRecord) -1;211650816734133
				TeeFilter(TimestampRecord) -1;211651816748213
			\end{lstlisting}
			
		\subsection{Detailed Introduction}		
			
	\section{Supporting Tools}
		\subsection{Replay Monitoring Logs}
		\subsection{Convert Monitoring Timestamps}
		\subsection{KAX Viz}
		\subsection{KAX Runner}
		
	\section{TSLib \& OPAD}
	