\chapter{Trace Analysis and Monitoring via AspectJ}\label{chap:aspectJ}

This chapter will show in Section \ref{sec:aspectJ:annotation} how
to use AspectJ to mark methods to be monitored with a simple annotation
in order to avoid the manual monitoring as seen in Chapter \ref{chap:example}
and \ref{chap:componentsMonitoring}. Once the methods are marked, the AspectJ-Weaver-Agent
will surround the calls with the necessary code during runtime, similar
to the hardcoded code used in Section \ref{sec:example:monitoring}.
An alternative solution will be shown as well in Section \ref{sec:aspectJ:fullweaving}
in which everything will be monitored - without exceptions. Both solutions
can be used to reconstruct the architecture and to perform a trace
analysis. The result of both will be diagrams similar to Figure \ref{fig:bookstore:classAndSequenceDiagrams}.

The idea of weaving the monitoring-code into the ``plain'' code
during compile-time seems to suggest itself, but in this chapter it
is only shown how to perform the so called load-time-weaving - the
weaving during runtime, which is way more flexible than the compile-time-weaving.

\section{AspectJ Annotations}\label{sec:aspectJ:annotation}

To weave the code via AspectJ with the \Kieker{}-Framework,
some new files are required, including the AspectJ-Agent and a configuration
file for AspectJ. Following figure shows the resulting directory tree
with the necessary files, based on the BookstoreApplication shown
in Chapter \ref{chap:example}.

\begin{figure}[H]
\begin{graybox}
\dirtree{%
.1 \DirInDirTree{example/}. %\DTcomment{The root directory of the project}.
.2 \DirInDirTree{build/}\DTcomment{Directory for the Java class files}.
.2 \DirInDirTree{META-INF/} \DTcomment{Directory for the configuration files}.
.3 \aopConfigFile. 
.2 \DirInDirTree{lib/} \DTcomment{Directory for the needed libraries}.
.3 \monitoringJar.
.3 \commonJar.
.3 \commonsLoggingJar.
.3 \aspectJWeaverJar.  
.2 \DirInDirTree{src/}\DTcomment{Directory for the source code files}.
.3 \DirInDirTree{bookstoreApplication/}.
.4 Bookstore.java.
.4 BookstoreStarter.java.
.4 Catalog.java.
.4 CRM.java.  
}
\end{graybox}

\caption{The new directory structure of the Bookstore application}
\end{figure}

The new jar-file \file{aspectjweaver-1.6.9.jar} can again be found
in the \dir{lib}-dir from the \Kieker{}-binaries, as well as the
configuration file \file{\aopConfigFile} can be found in the directory \dir{META-INF}.

Once the necessary file has been copied to the example-directory,
the sourcecode can be instrumented with the annotation \class{OperationExecutionMonitoringProbe}.
Listing \ref{listing:BookstoreAspectJ} shows how the annotation is
used.

\setJavaCodeListing
\lstinputlisting[caption=Bookstore.java, label=listing:BookstoreAspectJ]{\aspectJBookstoreApplicationDir/src/bookstoreTracing/Bookstore.java}

As an example all methods within the four sourcecode-files will be
annotated. It is possible to mark nearly every method with the annotation
- except constructors.
\TODO{Something is missing, I know...}
\section{Full Monitoring}\label{sec:aspectJ:fullweaving}
\TODO{Still missing something...}
