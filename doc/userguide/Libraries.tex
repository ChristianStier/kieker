\begin{center}
\begin{longtable}{|p{0.3\textwidth}|p{0.2\textwidth}|p{0.4\textwidth}|}
\hline 
Filename & License & Description\\
\hline
\hline 
aspectjrt-1.6.11.jar & EPL - v 1.0 & This jar-file contains the runtime library for AspectJ programs.\\
\hline 
aspectjtools-1.6.11.jar & EPL - v 1.0 & This package contains the tools (the AspectJ Compiler and Browser) for AspectJ.\\
\hline 
aspectjweaver-1.6.11.jar & EPL - v 1.0 & This jar contains the weaver-agent for the aspect-oriented-extension for Java named AspectJ.\\
\hline 
commons-cli-1.2.jar & ASL - v2.0 & Apache Commons CLI provides a simple API for working with the command line arguments and options.\\
\hline 
commons-logging-1.1.1.jar & ASL - v2.0 & Apache Commons Logging is a thin adapter allowing configurable bridging to other, well known logging systems.\\
\hline 
cxf-common-utilities-2.4.2.jar & ASL - v2.0 & This package contains different classes for Apache CXF.\\
\hline 
cxf-rt-bindings-soap-2.4.2.jar & ASL - v2.0 & This package contains necessary files to use Apache CXF as well with the Simple Object Access Protocol (SOAP).\\
\hline 
cxf-rt-core-2.4.2.jar & ASL - v2.0 & This library contains the Apache CXF Runtime Core.\\
\hline 
junit-4.10.jar & CPL - v 1.0 & This jar-file contains the necessary classes for the JUnit-tests, which can be used to test automatically Java classes.\\
\hline 
log4j-1.2.16.jar & ASL - v2.0 & Apache log4j is a framework for the logging of messages, errors and exceptions in Java applications.\\
\hline 
sigar-1.6.4.jar & ASL - v2.0 & Hyperic SIGAR (System Information Gatherer) provides a Java API to system inventory and monitoring data (Memory, CPU etc.). In addition to the Jar file, it is required to add corresponding platform-specific native libraries to the classpath, which can be downloaded from~\cite{HypericSigarWebsite}. Kieker's \dir{lib/} folder already includes such libraries for Linux/Windows for the x86~architecture (\file{libsigar-x86-linux.so} and \file{sigar-x86-winnt.[dll|lib]}.\\
\hline 
\end{longtable}
\label{tabular:libraries}
\end{center}
\begin{description}
\item[ASL] The Apache Software License
\item[CPL] Common Public License
\item[EPL] Eclipse Public License
\end{description}
