\begin{center}
\begin{longtable}{|p{0.4\textwidth}|p{0.5\textwidth}|}
\hline 
Filename & Description\\
\hline
\hline 
spring.jar & The spring framework delivers different methods and classes to make the handling with Java/Java EE easier.\\
\hline 
derby.jar & Apache Derby is a lightweight database written in Java which can also be used as an embedded database. This library contains the necessary drivers for the database as well as the database managment system itself.\\
\hline 
commons-pool-1.2.jar & Apache Commons Pool is an Object-pooling API supplying different interfaces and classes to create modular object pools.\\
\hline 
concurrent-1.3.4.jar & This library supplies different thread-safe classes for the enhanced development of multithreaded Java applications.\\
\hline 
junit-4.5.jar & This jar-file contains the necessary classes for the JUnit-tests, which can be used to test automatically Java classes.\\
\hline 
commons-logging-1.1.1.jar & Apache Commons Logging is a thin adapter allowing configurable bridging to other, well known logging systems.\\
\hline 
servlet-api.jar & The Java Servlet API supplies protocols to let applications respond for example to HTTP requests.\\
\hline 
aspectjweaver-1.6.9.jar & This jar contains the weaver-agent for the aspect-oriented-extension for Java named AspectJ.\\
\hline 
openjms-net-0.7.7-beta-1.jar & OpenJMS is an open source implementation of Sun Microsystems's Java Message Service API 1.1 Specification\\
\hline 
spring-web.jar & This library contains the web application context, multipart resolver, Struts support, JSF support and web utilities for the spring framework.\\
\hline 
commons-io-1.2.jar & Apache Commons-IO contains utility classes, stream implementations, file filters, and endian classes.\\
\hline 
spice-jndikit-1.2.jar & The JNDI Kit is a toolkit for the easy use of the so called Java Naming and Directory Interface.\\
\hline 
cxf-common-utilities-2.2.9.jar & This package contains different classes for Apache CXF.\\
\hline 
rabbitmq-client.jar & This library contains the client for the RabbitMQ messaging system.\\
\hline 
jndi-1.2.1.jar & The Java Naming and Directory Interface is an API which provides methods for multiple naming and directory services. It can be used for example to register disposed files in a network and to allow other part of a Java program to use them for RMI calls.\\
\hline 
cxf-rt-bindings-soap-2.2.9.jar & This package contains necessary files to use Apache CXF as well with the Simple Object Access Protocol (SOAP).\\
\hline 
log4j-1.2.15.jar & Apache log4j is a framework for the logging of messages, errors and exceptions in Java applications.\\
\hline 
cxf-api-2.2.9.jar & Apache CXF is an open source services framework.  \\
\hline 
cxf-rt-core-2.2.9.jar & This library contains the Apache CXF Runtime Core. \\
\hline 
aspectjtools-1.6.9.jar & This package contains the tools (the AspectJ Compiler and Browser) for AspectJ.\\
\hline 
jms-1.1.jar & Java Message Service is an API to send and receive messages within a client and to control so called Message Oriented Middleware (MOM).\\
\hline 
aspectjrt-1.6.9.jar & This jar-file contains the runtime library for AspectJ programs.\\
\hline 
servlet.jar & This package contains different classes for the work with servlets.\\
\hline 
mysql-connector-java-5.1.5-bin.jar & This library contains the drivers to connect from a Java application to a MySQL database system.\\
\hline 
openjms-common-0.7.7-beta-1.jar & OpenJMS is an open source implementation of Sun Microsystems's Java Message Service API 1.1 Specification\\
\hline 
jmc.jar & This library contains the Java Media Components which can be used for example for playing video content in Swing applications.\\
\hline 
Scenario.jar & This package provides scene graph functionality for Java.\\
\hline 
commons-cli-1.2.jar & Apache Commons CLI provides a simple API for working with the command line arguments and options.\\
\hline 
openjms-0.7.7-beta-1.jar & OpenJMS is an open source implementation of Sun Microsystems's Java Message Service API 1.1 Specification\\
\hline 
\end{longtable}
\label{tabular:libraries}
\end{center}
