% Set to english language and utf8.
\usepackage[english]{babel}
\usepackage[utf8]{inputenc}

% Some packages for symbols we need within the tutorial.
\usepackage{dingbat}
\usepackage{marvosym}

% For the sourcecode.
\usepackage{listings}

% Enable Jan's highlighting
% Usage:
% %% preamble
% \usepackage{listings} %% correct name?
\usepackage{lstide}
% \lstset{tabsize=2,captionpos=b,style=default,}
%
%
% %% main
%     \begin{lstlisting}[style=eclipse-java,gobble=6,caption={Simple micro-benchmark in Java}]
%       for (int i = 0; i < 1000; i++) {
%         tin = currentTime();
%         benchmarkedOperation // testtext
%         tout = currentTime();
%       }
%     \end{lstlisting}

% For the links etc.
\usepackage{hyperref} % avh: removed [pdfborder={0 0 0}]

% For the pdf-graphics.
\usepackage{graphicx}

% The steamroller tactics to fix figures and so on.
\usepackage{float}

% This is for tables which are to long to be shown on one page.
\usepackage{longtable}

% This package is for the directory tree structures
\usepackage{dirtree}
\renewcommand*\DTstylecomment{\footnotesize\it\rmfamily}
\renewcommand*\DTstyle{\footnotesize\sffamily}

% We need this package for some color within the document.
\usepackage{color}

\usepackage[sort,compress,numbers]{natbib} %round,authoryear

% compactitem, compactenum, ...
\usepackage{paralist}


% This is the package for the margin-nodes.
\usepackage[color=white, bordercolor=white]{todonotes}

\usepackage{amsfonts}
\usepackage{setspace}
\usepackage{ae,aecompl}

\usepackage[automark]{scrpage2}

\usepackage[margin=0.5cm,indention=0em,font={small},labelfont={sf,small},format=hang]{caption}

\usepackage[hang,sf]{subfigure}
\subfigcapmargin=1em

%\usepackage{scrhack}
